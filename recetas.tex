\documentclass{book}

\usepackage[utf8]{inputenc}
\usepackage[T1]{fontenc}
\usepackage{textcomp}
\usepackage[spanish]{babel}
\usepackage{hyperref}

\title{Recetas}
\author{Luis Mario Chaparro Jáquez}

\begin{document}
\maketitle
\tableofcontents

\newpage
\section{Sopa Miso}
\subsection*{Ingredientes:}
\begin{itemize}
	\item Caldo de pescado en polvo 
		\subitem Preferentemente bonito o hígado de bonito.
		\subitem Para una opción vegana usa caldo de alga.
	\item Vegetales al gusto.
		\subitem Cebollines, zanahorias, hongos, rábanos.
	\item Alga.
	\item Tofu.
	\item Pasta Miso.
\end{itemize}
\subsection*{Instrucciones:}
\begin{enumerate}
	\item Hierve todos los ingredientes con excepción de la pasta miso.
		Primero hierve los vegetales más duros, y agrega el resto poco a poco.
	\item Cuando todo esté suave agrega la pasta miso y mantén el calor sin hervir.
\end{enumerate}

\newpage
\section{Curry}
\textbf{2 porciones}
\subsection*{Ingredientes:}
\begin{itemize}
	\item \textbf{Aceite con alto punto de humo.}
	\item \textbf{Pollo o camarones o verdura (300g).}
		\subitem Yo recomiendo que el pollo tenga hueso (para que esté más sabroso), pero puedes hacerlo con pechuga, o con cualquierpieza de pollo deshuesada.
		\subitem Si usas camarones, que sean pelados (para que sea fácil de comer) y relativamente grandes para que no se encojan demasiado.
		\subitem En el caso de la verdura, escoje algo esponjoso como la berenjena.
	\item \textbf{Tomate mediano (1).}
	\item \textbf{Cebolla grande (1/2).}
	\item \textbf{Ajo (4 dientes).}
	\item \textbf{Pasta de gengibre y ajo (2 cdtas).}
	\item \textbf{Cúrcuma (1/3 cdta).}
	\item \textbf{Garam masala (3/4 cdta).}
	\item \textbf{Cilantro en polvo (2 cdtas).}
	\item \textbf{Chile en polvo (al gusto, yo recomiendo 1 cdta).}
	\item \textbf{Agua o leche de coco (150ml).}
\end{itemize}
Si usas leche de coco evita la pasta de gengibre y ajo. También toma en cuenta que la leche de coco quitara un poco del picor.
Si lo acompañas con arroz, asegúrate de cocinar el arroz desde antes.
\subsection*{Instrucciones:}
\begin{enumerate}
	\item Picar finamente la cebolla, ajo y tomate.
	\item Sofreir la cebolla.
	\item Cuando esté dorada la cebolla agregar el ajo y la pasta de gengibre y ajo.
		Freir hasta que el ajo dore.
	\item Agregar tomate y machacarlo un poco hasta que forme una pasta.
	\item Agregar todos los polvos agregando aceite o agua para que no se seque la mezcla.
\end{enumerate}
\textit{En este momento ya tienes la base, a la que llamamos curry.}
\begin{enumerate}
	\item Picar tu ingrediente principal en cubos del tamaño que quieras. El camarón depende del tamaño lo agregas cortado o sin cortar.
	\item Agregar pollo/camarón/verdura con agua (leche de coco) para que la mezcla sea más líquida y el ingrediente se marine.
	\item Baja la temperatura, tapa y dejar que el ingrediente se cocine.
\end{enumerate}

\subsection*{Cómo servir}
Servir acompañado de arroz o chapati (o ambos).

\newpage
\section{Lentejas de mamá}
\subsection*{Ingredientes:}
\begin{itemize}
	\item \textbf{Lentejas.}
	\item \textbf{Sal.}
	\item \textbf{Ajo.}
	\item \textbf{Cebolla.}
	\item \textbf{Tomate.}
	\item \textbf{Cilantro.}
	\item \textbf{Apio.}
	\item \textbf{Consomate.}
	\item \textbf{Azafrán.}
	\item \textbf{Queso.}
	\item \textbf{Limón.}
\end{itemize}
\subsection*{Instrucciones:}
\begin{enumerate}
	\item Cocer lentejas con sal y ajo hasta que estén suaves.
		Puedes dejar remojar unas cuantas horas antes para ablandar.
	\item Freir cebolla y tomate.
		Agregar a las lentejas cuando estén dorados y mezclar.
	\item Agregar apio, cilantro, consomate y azafrán.
	\item Servir con queso y limón.
\end{enumerate}

\newpage
\section{Ragù}
Inspirada en la receta de
\href{https://ricette.giallozafferano.it/Ragu-alla-bolognese.html}{Giallo Zafferano},
traducida y con comentarios.

\textbf{4 porciones}

\subsection*{Ingredientes:}
\begin{itemize}
	\item \textbf{Aceite extravirgen de oliva (1 cda).}
	\item \textbf{Pancetta (150g).} Si no encuentras pancetta el tocino también funciona, pero intenta conseguir tocino en trozo, no en tiras.
	\item \textbf{Zanahoria (50g).} Pelada.
	\item \textbf{Cebolla amarilla (50g).} Si no tienes cebolla amarilla, cebolla blanca está bien.
	\item \textbf{Apio (50g).}
	\item \textbf{Carne molida de res (300g).} Usa  carne molida con un nivel graso medio, no muy magra, no muy grasa.
	\item \textbf{Passata de tomate (300g).} Honestamente, no entiendo la diferencia entre la passata y el puré de tomate, para mí puedes poner puré \textit{sin sazonar} y no hay problema.
	\item \textbf{Vino tinto (100g).} Si tienes concentrado de vino tinto también funciona, sólo sigue las instrucciones de su empaque.
	\item \textbf{Caldo de verduras (al gusto/lo necesario).} Hazlo con tiempo o utiliza cubos de caldo de verduras, sigue las instrucciones del empaque en el último caso.
	\item \textbf{Pimienta negra (al gusto).}
	\item \textbf{Sal fina (al gusto).}
\end{itemize}
\subsection*{Instrucciones:}
Puedes desde un inicio cortar tus vegetales y pancetta (o tocino), para no peder tiempo, aunque tampoco es muy tardado.
Por supuesto, estoy asumiendo que ya lavaste tus vegetales.
\begin{enumerate}
	\item Corta tu pancetta en cubos pequeños, no es necesario ser muy preciso.
	\item En un sartén hondo ya caliente a temperatura media alta, añade aceite de oliva y la pancetta, déjala dorar por un tiempo.
	\item Mientras la pancetta se dora, corta tus vegetales (zanahoria, cebolla y apio) en cubos tan grandes como cortaste la pancetta.
	\item En cuanto la pancetta se vea bien dorada, añade tus vegetales al sartén.
	\item Mezcla y dora por 5 minutos.
	\item Añade la carne molida sube la flama un poquito y dórala sin prisa. La razón para subir la flama es para que el calor no se pierda al agregar toda la carne que estará a temperatura baja, para nada es para que la carne se cocine más rápidamente.
	\item Una vez dorada la carne añade el vino tinto.
	\item Cuando casi todo el vino se haya evaporado, añade la passata de tomate e incorpora todo.
	\item Añade un poco de caldo de verduras, dos cucharones está bien.
	\item Tapa el sartén pero deja una abertura, si la tapa tiene salida para el vapor con eso hay.
	\item En principio a este punto, el ragù debe cocinarse por 2 horas a temperatura baja y revisando cada 20 minutos, agregando caldo de verduras para evitar que se seque. La verdad yo suelo cocinarlo por unos 45 minutos y el resultado es bastante bueno.
	\item También este es el momento en que agregarás sal y pimienta, ajustando poco a poco a tu gusto.
\end{enumerate}

\subsection*{Cómo servir}
El ragù es la salsa con la que se hace la lasagna, entonces puedes hacer eso, claro que para ello necesitas un poquito más de ingredientes (béchamel, queso parmesano, etc.).
Una opción más rápida es cocinar un poco de pasta, yo recomiendo tagliatele, que es la forma tradicional, pero cualquier pasta en fideo plano va perfecto, ya si no tienes de otra spaghetti esta bien.
No sé decir cuánta pasta usar, pero si es tagliatele, que suele venir en nidos, dos nidos por cada 150g de ragù queda buenísimo. 
Si tienes tiempo de pesar cosas, nomás asegurate de agregar tanta pasta como sea posible para que al mezclar con la salsa todo quede cubierto.

\end{document}
