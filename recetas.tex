\documentclass[a4paper]{article}

\usepackage[utf8]{inputenc}
\usepackage[T1]{fontenc}
\usepackage{textcomp}
\usepackage[spanish]{babel}

\title{Recetas}
\author{Luis Mario Chaparro Jáquez}

\begin{document}
	\maketitle
	\tableofcontents

	\newpage
	\section{Sopa Miso}
	\subsection{Ingredientes:}
	\begin{itemize}
		\item Caldo de pescado en polvo 
			\subitem Preferentemente bonito o hígado de bonito.
			\subitem Para una opción vegana usa caldo de alga.
		\item Vegetales al gusto.
			\subitem Cebollines, zanahorias, hongos, rábanos.
		\item Alga.
		\item Tofu.
		\item Pasta Miso.
	\end{itemize}
	\subsection{Instrucciones:}
	\begin{enumerate}
		\item Hierve todos los ingredientes con excepción de la pasta miso.
			Primero hierve los vegetales más duros, y agrega el resto poco a poco.
		\item Cuando todo esté suave agrega la pasta miso y mantén el calor sin hervir.
	\end{enumerate}

	\newpage
	\section{Curry}
	\subsection{Ingredientes:}
	\begin{itemize}
		\item Pollo/camarones/verdura (300g)
		\item Tomate (1)
		\item Cebolla grande (1/2)
		\item Ajo (4 dientes)
		\item Pasta de gengibre y ajo (2 cdtas)
		\item Cúrcuma (1/3 cdta)
		\item Garam masala (3/4 cdta)
		\item Cilantro en polvo (2 cdtas)
		\item Chile en polvo (al gusto, 1/2 cdta)
		\item Agua o leche de coco.
                \item Para acompañar: Arroz blanco, chapati o ambos (tortillas de harina).
	\end{itemize}
	\textit{Si usas leche de coco evita la pasta de gengibre y ajo.}
	\subsection{Instrucciones:}
	\begin{enumerate}
		\item Picar finamente cebolla, ajo y tomate.
		\item Sofreir cebolla.
		\item Cuando esté dorada la cebolla agregar el ajo y la pasta de gengibre y ajo.
			Freir hasta que el ajo dore.
		\item Agregar tomate y machacarlo un poco hasta que forme pasta.
		\item Agregar todos los polvos agregando aceite para que no se seque la mezcla.
	\end{enumerate}
	\textit{En este momento ya tienes la base, a la que llamamos curry.}
	\begin{enumerate}
		\item Picar pollo/camarones/verdura en cubos del tamaño que quieras.
			El camarón depende del tamaño lo agregas cortado o sin cortar.
		\item Agregar pollo/camarón/verdura con agua (leche de coco) para que la mezcla sea más líquida y el ingrediente se marine.
		\item Tapar y dejar que el ingrediente se cocine.
	\end{enumerate}
	\textit{Servir acompañado de arroz o chapati (o ambos)}

	\newpage
	\section{Lentejas}
	\subsection{Ingredientes:}
	\begin{itemize}
		\item Lentejas.
		\item Sal.
		\item Ajo.
		\item Cebolla.
		\item Tomate.
		\item Cilantro.
		\item Apio.
		\item Consomate.
		\item Azafrán.
		\item Queso.
		\item Limón.
	\end{itemize}
	\subsection{Instrucciones:}
	\begin{enumerate}
		\item Cocer lentejas con sal y ajo hasta que estén suaves.
			Puedes dejar remojar unas cuantas horas antes para ablandar.
		\item Freir cebolla y tomate.
			Agregar a las lentejas cuando estén dorados y mezclar.
		\item Agregar apio, cilantro, consomate y azafrán.
		\item Servir con queso y limón.
	\end{enumerate}
\end{document}
